%%%%%%%%%%%%%%%%%%%%%%%%%%%%%%%%%%%%%%%%%
% Medium Length Professional CV
% LaTeX Template
% Version 2.0 (8/5/13)
%
% This template has been downloaded from:
% http://www.LaTeXTemplates.com
%
% Original author:
% Rishi Shah 
%
% Important note:
% This template requires the resume.cls file to be in the same directory as the
% .tex file. The resume.cls file provides the resume style used for structuring the
% document.
%
%%%%%%%%%%%%%%%%%%%%%%%%%%%%%%%%%%%%%%%%%

%----------------------------------------------------------------------------------------
%	PACKAGES AND OTHER DOCUMENT CONFIGURATIONS
%----------------------------------------------------------------------------------------

\documentclass{resume} % Use the custom resume.cls style

\usepackage[left=0.75in,top=0.6in,right=0.75in,bottom=0.6in]{geometry} % Document margins
\usepackage{url}
\newcommand{\tab}[1]{\hspace{.2667\textwidth}\rlap{#1}}
\newcommand{\itab}[1]{\hspace{0em}\rlap{#1}}
\name{W\MakeLowercase{illiam} L\MakeLowercase{ee}} % Your name
\address{500 College Ave, Swarthmore, PA} % Your address
%\address{123 Pleasant Lane \\ City, State 12345} % Your secondary addess (optional)
\address{\url{cs.swarthmore.edu/~wlee1} \\ wlee1@swarthmore.edu} % Your phone number and email

\begin{document}

%----------------------------------------------------------------------------------------
%	EDUCATION SECTION
%----------------------------------------------------------------------------------------

\begin{rSection}{Education}
	{ \textbf{Swarthmore College} (Computer Science)} \hfill {\em 2020 expected (3.8+)} 
\end{rSection}

%----------------------------------------------------------------------------------------
%	WORK EXPERIENCE SECTION
%----------------------------------------------------------------------------------------
\begin{rSection}{Work Experience}
	{
		\begin{rSubsection}{Behavior Prediction,  Waymo}{\em Summer 2019}{}{}
			--- Machine learning for self-driving cars at Waymo (formerly Google's self-driving car project).
		\end{rSubsection}
	}
	{
		\begin{rSubsection}{Search Relevance, Salesforce}{\em Summer 2018}{}{}
			--- Redesigned the search metrics pipeline for Salesforce Search Cloud using Splunk and Hadoop. \\
			--- Data analysis on terabyte-scale logs to accurately measure Salesforce Search's performance.
		\end{rSubsection}
	
	}
	{
		\begin{rSubsection}{Laboratory for Advanced Sensing, NASA Ames}{\em Summers 2016 and Summer 2017}{}{}
			--- Designed and implemented a computer vision alignment pipeline to fully georectify 4k UAV footage onto satellite imagery. \\
			--- Implemented a sliding window CNN using Keras to classify coral reef morphology. \\
			--- Developed a entropy-based adaptive gaussian blur module for NASA's stereogammetry suite.
		\end{rSubsection}
	}

\end{rSection}  % End Work Experience.

\begin{rSection}{Academic Experience}
	{\textbf{St. Anne's College, University of Oxford} Visiting student at the University of Oxford for the 2018--2019 academic year in the mathematics and computer science departments. Includes graduate level coursework in Advanced Machine Learning and Randomized Algorithms.}

	{\textbf{Research Assistant, Biomedical Machine Learning Lab} Research assistant working with Professor Ameet Soni on weakly supervised learning for Chest Xrays.}
\end{rSection}  % End Academic Experience.

%--------------------------------------------------------------------------------
%    Projects And Seminars
%-----------------------------------------------------------------------------------------------
\begin{rSection}{Projects}
	{\textbf{Capturing Population Events Using HMMs}
	\\Analyzed genomic sequence data from different human populations in the 1000 Genomes Project. Combined multiple projects from a semester-long Bioinformatics class to create an end-to-end genomic pipeline to convert raw sequence data to population-size estimates using TMRCA. Successfully captured the out-of-Africa bottleneck; concluded that smaller-scale population events require more data. \\}
\\
	{\textbf{Weight Uncertainty in Neural Networks}\\
	Replicated and extended the results found in \textit{Weight Uncertainty in Neural Networks (Blundell et al., 2015)}.}

\end{rSection}
%----------------------------------------------------------------------------------------
%	TECHNICAL STRENGTHS SECTION
%----------------------------------------------------------------------------------------


\begin{rSection}{Coursework}  %  Begin Coursework.
	{
		\begin{tabular}{ @{} >{\bfseries}l @{\hspace{6ex}} l }
			University of Oxford \ & Advanced Machine Learning, Probability \& Computing, \\ \ & Probability, Artificial Intelligence \\

			Swarthmore College \ & Machine Learning, Computer Vision, Bioinformatics \\
		\end{tabular}
	}
\end{rSection}  % End Coursework.

\begin{rSection}{Skills}
	{Python, PyTorch, C++, Java, OpenCV, Sklearn, Hadoop/Splunk}
\end{rSection}  % End Skills.

\end{document}
